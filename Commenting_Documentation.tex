\documentclass[StyleGuide.tex]{subfiles}

\begin{document}

\chapter{Commenting \& Documentation}\label{commenting-documentation}

\section{Comments}\label{comments}

Comments should be included where necessary to give a better
understanding of code. When writing longer comments, follow Strunk \&
White's \emph{The Elements of Style} for English grammar guidelines.

When writing dates and times, use the ISO 8601 standard. That is, for
dates, YYYY-MM-DD; for times, hh:mm:ss; and, for date-times,
YYYY-MM-DDThh:mm:ss.

Comments should be included that give detailed descriptions of all
functions, classes, and instance variables.

\section{Program Documentation}\label{program-documentation}

It is essential to provide a brief overview of your program. Such an
overview should be placed in a file named \texttt{README.md}, and
formatted using Markdown (a markup language designed to create
nicely-formatted, but still easily parsed, plain-text files). This file
should contain an overview of the program's purpose, its authors, and
relavent copyright information. It should also include directions on
where to find more extensive documentation, for both users and
developers. However, detailed documentation on the program's
functionality should \emph{not} be included in this file.

In addition, it is prudent to maintain a \texttt{CHANGELOG} (or, if
preferred, \texttt{CHANGELOG.md}) file, in which a brief overview of the
changes made between each release of the program are listed. Note that
the changes listed here should be only the major differences between
versions; all minor changes and modifications should be recorded using
some form of version control, such as Git or SVN.

\section{Function Documentation}\label{function-documentation}

DocBlock-style comments should be provided for \emph{every} function.
DocBlock comments should be formatted using the \texttt{/**\ ...\ */}
system, as illustrated in \autoref{lst:doxygen-example}.

The most commonly used DocBlock interpreter for C documentation is
\href{http://www.stack.nl/~dimitri/doxygen/}{Doxgen}. It supports export
into a multitude of formats and supports a large number of tags. For a
reference of what tags are supported, see the
\href{http://www.stack.nl/~dimitri/doxygen/manual/commands.html}{Special
Commands} section of the
\href{http://www.stack.nl/~dimitri/doxygen/manual/index.html}{Doxygen
Manual}.

Author and date (or \texttt{@since}) should \emph{always} be given for a
file DocBlock. If using version numbering for the project, the version
tag (showing the current version number, not that at which the file was
introduced) should also \emph{always} appear in the file DocBlock.

At a minimum, functions should \emph{always} have the \texttt{@date} (or
\texttt{@since}) tags. If multiple authors have contributed to the same
file, then each function's block \emph{must} include an \texttt{@author}
tag, as well. (If not given, the author is to be assumed to be the same
as the file author.)

Functions should also have \texttt{@param} and \texttt{@return} tags
where applicable. Parameter tags should specify the parameter direction
(\texttt{{[}in{]}} or \texttt{{[}out{]}}, as seen in the example below).
The description portion of these tags should be located on the next line
(after the tag and variable name) and indented one level past the tag
itself.

If absolutely necessary to use, global variables should have a DocBlock
comment to explain their purpose. Macros should be described in a block
somewhere near the beginning of the file (though not in the file's
description/header block).

Description text should be written in sentences and be punctuated with a
period. Variable and \texttt{@param}/\texttt{@return} value
descriptions, however, should be kept brief and not written as full
sentences, unless absolutely necessary. (The first letter of these
descriptions, though, should still be capitalized.)

A blank line should be placed after each paragraph of explanation;
between the explanation and author, date, and version tags; and between
those tags and the \texttt{@param} and \texttt{@return} tags. (Note,
however, that there should \emph{not} be a blank line between
\texttt{@param} and \texttt{@return} tags.)

\begin{code}[caption=Example of Doxygen comments.,label=lst:doxygen-example]
/**
 * This is an explanation of a function.
 *
 * @author Ethan Ruffing <ruffinge@gmail.com>
 * @since 2014-08-06
 *
 * @param[in] varName
 *     A brief description of the parameter varName
 * @param[out] outName
 *     A brief description of the output outName
 * @return
 *     A brief description of the function's return value
 */
char doSomething(int varName, int *outName) {
    *outName = varName + 1;
    return varName % 2 == 0 ? 'y' : 'n';
}
\end{code}

\end{document}
