\documentclass[StyleGuide.tex]{subfiles}

\begin{document}

\chapter{File \& Program Organization}\label{ch:file-program-organization}

\section{File Format}\label{sec:file-format}

Source code should be written so that no line exceeds eighty characters
in length.

A header block should be placed at the top of each file giving the
author, the date of creation, and, if applicable, the version number.
This header block should also include a copyright notice and license
header if applicable.

File names should be clear and concise so that someone can tell the
general purpose of the file without having to open and read it or its
header block.

\section{File Structure}\label{sec:file-structure}

Function \emph{declarations} should always be located within a header
(\texttt{*.h}) file, while function \emph{definitions} should be located
in a source (\texttt{*.c}) file. If choosing to place documentation for
the function in only one of these locations, it is best to place it in
the header file, as the header file is intended to contain a high-level
overview of each function.

\subsection{Header Files}\label{subsec:header-files}

Header files should be designed so as to prevent problems from being included
multiple times. This is to be achieved through the use of the preprocessor macros
\texttt{\#ifndef} and \texttt{\#endif}, as shown in \autoref{lst:header-layout}.
It is helpful to place the name of the defined variable in a comment at the
\texttt{\#endif}, as well.

\begin{code}[caption=Example of header file layout.,label=lst:header-layout]
/**
 * @file example.h
 */
 
#include <stdio.h>
#include <stdlib.h>

#ifndef EXAMPLE_H_
#define EXAMPLE_H_

// User code here

#endif /* EXAMPLE_H_ */
\end{code}

\section{Program Structure}\label{sec:program-structure}

Every effort should be made to avoid the use of global variables.
Instead, pre-processor macros should be used for constants, and
variables should be passed and returned between functions.

\end{document}
