\documentclass[StyleGuide.tex]{subfiles}

\begin{document}

\chapter{Basic Formatting}\label{basic-formatting}

C source files should be written according the the Kernighan and Ritchie
(K \& R) style. Furthermore, source files should be written according to
the C11 (ISO/IEC 9899:2011) standard. Extensive documentation on both of
these can be found online.

\section{Naming}\label{naming}

In general, names should be concise, but long enough to understand
immediately.

The following rules are non-negotiable with regards to the case used for
names.

\subsection{Variables and Functions}\label{variables-and-functions}

All variables and functions should be named in
\texttt{headlessCamelCase}.

\subsection{Constants and Macros}\label{constants-and-macros}

Constants and pre-processor macros should be named in
\texttt{SCREAMING\_SNAKE\_CASE}.

\section{Whitespace}\label{whitespace}

Whitespace helps tremendously to improve readability. The following are
basic guidelines, but can be expanded on to further improve readability.

\subsection{Indentation}\label{indentation}

There seems to be an ongoing war regarding indentation in programming.
For the purposes of this style guide, we will require the use of hard
tabs (an actual \texttt{\textbackslash{}t} character). IDEs and text
editors should be set to use a tab width of four, and alignment of items
such as comments should be based on this assumption.

There is one exception to this rule (which was alluded to previously):
When formatting block comments, all text within the block should be
aligned using spaces in order to guarantee readable documentation
everywhere. (NOTE, however, that the comment block itself should be
indented using tabs.)

\subsection{Blank Lines}\label{blank-lines}

One blank line should always be used in the following locations:

\begin{itemize}
\item
  Between functions
\item
  Between a local variable in a function and its first statement
\item
  Before a block or single-line comment
\item
  Between logically separate sections of code within a function
\end{itemize}

\subsection{Blank Spaces}\label{blank-spaces}

Blank spaces should be used in the following locations in order to enhance
readability (an example is shown in \autoref{lst:spacing-example}):

\begin{itemize}
\item
  Between a keyword and a parenthesis. Note that a blank space should
  \emph{not} be used between a function and its opening parenthesis. Also,
  there should not be any blank space following an opening parenthesis
  or preceding a closing parenthesis.
\item
  Before an opening brace (see above example)
\item
  Between binary and ternary operators and their operands
\item
  \emph{Not} between unary operators and their operands
\item
  After each semicolon in a \texttt{for} loop
\end{itemize}

\begin{code}[caption=Spacing example, label=lst:spacing-example]
void example(int a) {
	int i, b = 2, c;
	for (i = 0; i < n; i++) {
		c = a + b;
		c = a ? b - 1 : b + 1;
	}
}
\end{code}

\subsection{Braces}\label{braces}

\begin{itemize}
\item
  All loops, conditionals, and other such structures where braces are
  functionally ``optional'' \emph{must} use braces, even if their
  contents is a single line.
\item
  Opening braces should always be on their own line, aligned on the
  column of the item they are for.
\item
  Close braces should always be on their one line, aligned horizontally
  with their open statement.
\end{itemize}

\end{document}
