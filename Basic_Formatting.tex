\documentclass[StyleGuide.tex]{subfiles}

\begin{document}

\chapter{Basic Formatting}\label{ch:basic-formatting}

C source files should be written according the the Kernighan and Ritchie
(K \& R) style. Furthermore, source files should be written according to
the C11 (ISO/IEC 9899:2011) standard. Extensive documentation on both of
these can be found online.

\section{Naming}\label{sec:naming}

In general, names should be concise, but long enough to understand
immediately.

The following rules are non-negotiable with regards to the case used for
names.

\subsection{Variables and Functions}\label{subsec:variables-and-functions}

All variables and functions should be named in
\texttt{headlessCamelCase}.

\subsection{Constants and Macros}\label{subsec:constants-and-macros}

Constants and pre-processor macros should be named in
\texttt{SCREAMING\_SNAKE\_CASE}.

\section{Whitespace and Layout}\label{sec:whitespace}

Whitespace helps tremendously to improve readability. The following are
basic guidelines, but can be expanded on to further improve readability.
Furthermore, the layout of a source file can have a significant impact
on readability. The layout includes things such as line length, alignment,
etc.

\subsection{Indentation}\label{subsec:indentation}

There seems to be an ongoing war regarding indentation in programming. Many
developers prefer hard tabs, while others prefer spaces. This style guide
recommends a combination of the two: tabs for indentation, and spaces for
alignment.

Hard tabs (an actual \texttt{\textbackslash{}t} character) shoudl be used to
indent your code (i.e., within a function, all code is indented one tab
character past the level of the function definition). IDEs and text editors
should be set to use a tab width of four, and alignment of items such as
comments should be based on this assumption.

However, when aligning text within the same indentation level (such as aligning
lines within a block comment, or aligning function arguments spanning multiple
lines), spaces should be used to ensure consistent alignment. Tab characters
should be used to achieve the appropriate indentation, followed by spaces to
align text.

\subsection{Line Length}\label{subsec:line-length}
Lines should never be permitted to exceed 80 characters in length. (When
calculating a line's length, treat a tab character as eight characters.)

Be consistent and sane in your choice of where a line is broken: always break
before an operator (e.g., break before a '+'), never break before a comma, etc.

Additionally, when breaking a line, make sure the continuation (the next line)
is indented one level past the original line, as shown in
\autoref{lst:simple-break-example}.

\begin{code}[caption=Simple line break example, label=lst:simple-break-example]
callfunction(many, many, many, arguments
    that, must, be, continued, on, the, next,
    line);
\end{code}

When breaking a long line with repeated components (such as a long list of additions),
align the repeated components to the same indentation level (e.g., align all addends
in a broken addition statement). An example of this is shown in
\autoref{lst:align-break-example}.

\begin{code}[caption=Aligned line break example, label=lst:align-break-example]
int sum = addend1
        + addend2
        + addend3
        + addend4
        + addend5;
\end{code}

\subsection{Blank Lines}\label{subsec:blank-lines}

One blank line should always be used in the following locations:

\begin{itemize}
\item
  Between functions
\item
  Between a local variable in a function and its first statement
\item
  Before a block or single-line comment
\item
  Between logically separate sections of code within a function
\end{itemize}

\subsection{Blank Spaces}\label{subsec:blank-spaces}

Blank spaces should be used in the following locations in order to enhance
readability (an example is shown in \autoref{lst:spacing-example}):

\begin{itemize}
\item
  Between a keyword and a parenthesis. Note that a blank space should
  \emph{not} be used between a function and its opening parenthesis. Also,
  there should not be any blank space following an opening parenthesis
  or preceding a closing parenthesis.
\item
  Before an opening brace (see above example)
\item
  Between binary and ternary operators and their operands
\item
  \emph{Not} between unary operators and their operands
\item
  After each semicolon in a \texttt{for} loop
\end{itemize}

\begin{code}[caption=Spacing example, label=lst:spacing-example]
void example(int a) {
	int i, b = 2, c;
	for (i = 0; i < n; i++) {
		c = a + b;
		c = a ? b - 1 : b + 1;
	}
}
\end{code}

\subsection{Braces}\label{subsec:braces}

\begin{itemize}
\item
  All loops, conditionals, and other such structures where braces are
  functionally ``optional'' \emph{must} use braces, even if their
  contents is a single line.
\item
  Opening braces should always be on their own line, aligned on the
  column of the item they are for.
\item
  Close braces should always be on their one line, aligned horizontally
  with their open statement.
\end{itemize}

\end{document}
