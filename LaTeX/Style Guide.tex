%! program = pdflatex

\documentclass[12pt,letter]{memoir} % for a long document
%\documentclass[12pt,letter,article]{memoir} % for a short document

% See the ``Memoir customise'' template for some common customisations
% Don't forget to read the Memoir manual: memman.pdf

\setlength{\parindent}{0pt}
\nonzeroparskip

\title{Personal Style Guide}
\author{Ethan Ruffing}
\date{Updated December 6, 2014} % Delete this line to display the current date

\frenchspacing

\usepackage{hyperref}

\usepackage{listings}
\usepackage{color}

\definecolor{dkgreen}{rgb}{0,0.6,0}
\definecolor{gray}{rgb}{0.5,0.5,0.5}
\definecolor{mauve}{rgb}{0.58,0,0.82}

\lstset{frame=tb,
  language=Java,
  aboveskip=3mm,
  belowskip=3mm,
  showstringspaces=false,
  basicstyle={\footnotesize\ttfamily},
  numbers=left,
%  numberstyle=\tiny\color{gray},
%  keywordstyle=\color{blue},
%  commentstyle=\color{dkgreen},
%  stringstyle=\color{mauve},
  %breaklines=false,
  %breakatwhitespace=true
%  tabsize=3
}

\copypagestyle{plainnotice}{plain}
\makeoddfoot{plainnotice}{}{\tiny Copyright \copyright \ 2014 Ethan Ruffing. All
	rights reserved. \normalsize}{}


%%% BEGIN DOCUMENT
\begin{document}

\maketitle
\thispagestyle{plainnotice}


\clearpage\mbox{}\thispagestyle{empty}\clearpage

% the asterisk means that the contents itself isn't put into the ToC
\tableofcontents*

\newpage
\thispagestyle{empty}
\mbox{}

\chapter*{Introduction}
In experience of writing programs and code, it has come to my attention that it
is beneficial to have predetermined conventions of coding style. Furthermore, I
have learned that these conventions must necessarily vary from one programming
language to the next due to the widely varying uses and though processes behind
each programming environment.

Therefore, I have decided to write this personal style guide as a reference for
myself, so that, in the future, I will have an easier time keeping track of
which conventions to use when.

This guide is organized in a hierarchical structure, first by programming
paradigm, then by programming language, and, finally, by feature. Any section of
this guide is intended to inherit the instructions of those sections above it,
and all are intended to inherit those instructions in
Chapter~\ref{chap:everywhere}:~\nameref{chap:everywhere}.

\chapter{Everywhere}\label{chap:everywhere}
	\section{File Organization}
		Source code should be written so that no line exceeds eighty characters
		in length.

		A header block should be placed at the top of each file giving the
		author, the date of creation, and, if applicable, the version number.
		This header block should also include a copyright notice and license
		header if applicable.

		File names should be clear and concise so that someone can tell the
		general purpose of the file without having to open and read it or its
		header block.

	\section{Commenting}
		Comments should be included where necessary to give a better
		understanding of code. When writing longer comments, follow Strunk \&
		White's \emph{The Elements of Style} for English grammar guidelines.

		When writing dates and times, use the ISO 8601 standard. That is, for
		dates, YYYY-MM-DD; for times, hh:mm:ss; and, for date-times,
		YYYY-MM-DDThh:mm:ss.

	\section{Indentation}
		There seems to be an ongoing war regarding indentation in programming.
		For the purposes of this style guide, we will require the use of hard
		tabs (an actual `\texttt{{\textbackslash}t}' character). IDE's and text
		editors should be set to use a tab width of four, and alignment of
		things such as comments should be based on this assumption.

		\subsection{Exceptions}
			There are two exceptions to this rule.

			First, when formatting block comments, all text within the block
			should be aligned using spaces in order to guarantee readable
			documentation everywhere. (NOTE, however, that the comment block
			itself should be indented using tabs.)

			Second, in languages that are whitespace-sensitive (such as Python)
			use spaces for all indentation.

			In both of these exceptions, when indenting using spaces, indent
			each level at four spaces past the last.

\chapter{Object-Oriented Programming}
	\section{All Languages}
		\subsection{Commenting}
			Comments should be included that give detailed descriptions of all
			functions, classes, and instance variables.

		\subsubsection{Documentation}
			When possible, DocBlock-style comments should be provided for
			\emph{every} class and method (public or private). DocBlock comments
			should be formatted using the \texttt{/** ... */} system, as
			illustrated in Figure \ref{lst:docblock-oop} on page
			\pageref{lst:docblock-oop}.

			Note that the available tags vary between
			interpreters for these comments. Here are some of the interpreters:
			\begin{itemize}
				\item \href{http://www.oracle.com/technetwork/java/javase/documentation/index-jsp-135444.html}{JavaDoc}
					comes with the Java Development Kit, and is used almost
					universally for Java documentation.
				\item \href{http://www.stack.nl/~dimitri/doxygen/}{Doxygen}
					supports many languages and is the most commonly used tool
					for C and C++ documentation.
				\item \href{http://apigen.org/}{ApiGen} (based on
					\href{http://www.phpdoc.org/}{phpDocumentor}) works well for
					PHP documentation and has built-in support in NetBeans.
					\href{http://www.sitepoint.com/generate-documentation-with-apigen/}{This blog post}
					at sitepoint.com can act as a good reference for the tags
					that ApiGen supports.
				\item \href{http://usejsdoc.org}{JSDoc} is a very full-featured
					system for documenting JavaScript source code.
			\end{itemize}

			Author and date (since) should \emph{always} be given for a class
			DocBlock. If using version numbering for the project, the version
			tag should also \emph{always} appear in the class DocBlock.

			At a minimum, functions/methods should \emph{always} have the date
			(since) tag. If multiple authors have contributed to the same class,
			they should also include the author tag. (If not given, the author
			is to be assumed to be the same as the class author.)

			Functions/methods should also have \texttt{@param} and
			\texttt{@return} tags where applicable. Any method other than a
			simple getter/setter (one which does not operate on the variable,
			but merely sets or returns its value) must have a description of its
			functionality. The description portion of these tags should be
			located on the next line (after the tag and variable name) and
			indented one level past the tag itself.

			Class and instance variables and fields should have a DocBlock
			comment to explain their purpose.

			Description text should be written in sentences and be punctuated
			with a period. Variable and \texttt{@param}/\texttt{@return} value
			descriptions, however, should be kept brief and not written as full
			sentences, unless absolutely necessary. (The first letter of these
			descriptions, though, should still be capitalized.)

			A blank line should be placed after each paragraph of explanation;
			between the explanation and author, date, and version tags; and
			between those tags and the param and return tags.
			\begin{figure}[h!]
				\label{lst:docblock-oop}
				\caption{Example DocBlock-style Commenting}
				\lstset{language=Java}
				\begin{lstlisting}
/**
 * This is an explanation of a function.
 *
 * @author Ethan Ruffing <ruffinge@gmail.com>
 * @since 2014-08-06
 *
 * @param varName A brief description of the parameter varName
 * @return A brief description of the function's return value
 */
				\end{lstlisting}
			\end{figure}
		\subsection{Naming}
			In general, names should be concise, but long enough to understand
			immediately.
			\subsubsection{Classes}
				Classes and interfaces should be titled in \texttt{CamelCase},
				with a name that reflects the real-world object which the class
				is intended to model.
			\subsubsection{Variables}
				All variables should be named in \texttt{headlessCamelCase}.
			\subsubsection{Constants}
				Constants (including enum entries) should be named in
				\texttt{UPPER\_CASE}.
		\subsection{Whitespace}
			Whitespace helps tremendously to improve readability. The following
			are basic guidelines, but can be expanded on to further improve
			readability.
			\subsubsection{Blank Lines}
				One blank line should always be used in the following locations:
				\begin{itemize}
					\item Between methods
					\item Between a local variable in a method and its first
						statement
					\item Before a block or single-line comment
					\item Between logically separate sections of code within a
						method
				\end{itemize}
			\subsubsection{Blank Spaces}
				Blank spaces should be used in the following locations:
				\begin{itemize}
					\item Between a keyword and a parenthesis. Note that a blank
						space should \emph{not} be used between a method and its
						opening parenthesis. Also, there should not be any blank
						space following an opening parenthesis or preceding a
						closing parenthesis. Example:
						\begin{lstlisting}
while (true) {
    System.out.println(testVar);
	...
}
						\end{lstlisting}
					\item Before an opening brace (see above example)
					\item Between binary and ternary operators and their
						operands
					\item \emph{Not} between unary operators and their operands
					\item After each semicolon in a \texttt{for} loop (for
						example, \texttt{for (int i = 0; i < n; i++) \{\}})
				\end{itemize}
		\subsection{Braces}
			All loops, conditionals, and other such structures where braces can
			be ``optional'' \emph{must} use braces, even if their contents is a
			single line.
		\subsection{Error Handling}
			There are multiple existing systems of error handling. For example,
			in traditional C, errors are passed via return values as integers.
			In some places, errors are passed as simple booleans (i.e.,
			\texttt{false} means an error occurred).

			Nearly all object-oriented programming languages, however, support
			the much more sophisticated system of throwing exceptions. This is
			the method that should be used whenever possible for error handling.
			It is a system that fits much more naturally into the paradigm of
			object-oriented programming and allows for much more detailed
			reports and, therefore, easier debugging.

			If a library is using some other form of error handling, it would be
			prudent to incorporate a wrapper of some form that will allow your
			program to throw an the error as an exception.

			Exceptions, when appropriate, should be thrown as high as possible.
			They should not be caught and ignored in a private function. Rather,
			they should make it as high as a static method (for example, in
			Java, \texttt{main()}), or the primary operating method of your
			program.
	\section{Java}
		Unless otherwise specified here, follow the guidelines of Oracle's
		\href{http://www.oracle.com/technetwork/java/codeconvtoc-136057.html}{Java Coding Style}.
		\subsection{Braces}
			\begin{itemize}
				\item Opening braces should always be on the same line as their
					parent.
				\item Close braces should always be on their own line, aligned
					horizontally with their open statement.
			\end{itemize}
		\subsection{File Contents}
			Each file should contain only one top-level class, interface, enum,
			etc. The file should be named identically to that top-level item,
			with the file extension \texttt{*.java}. For example, if a file
			contains \texttt{class HelloWorld}, it should be named
			\texttt{HelloWorld.java}.
	\section{PHP}
		Unless otherwise specified here, follow the guidelines of the PHP Pear
		\href{http://pear.php.net/manual/en/standards.php}{Coding Standards}.
		\subsection{Braces}
			\begin{itemize}
				\item Opening braces for classes and functions should always be
					on their own line, aligned on the column of the item they
					are for.
				\item Opening braces for loops and conditionals should always be
					on the same line as their parent.
				\item Close braces should always be on their own line, aligned
					horizontally with their open statement.
			\end{itemize}
	\section{JavaScript}
		Unless otherwise specified here, follow the guidelines of the
		\href{http://google-styleguide.googlecode.com/svn/trunk/javascriptguide.xml}{Google JavaScript Style Guide}
		\subsection{Braces}
			\begin{itemize}
				\item Opening braces should always be on the same line as their
					parent.
				\item Close braces should always be on their own line, aligned
					horizontally with their open statement.
			\end{itemize}
		\subsection{Semicolons}
			In JavaScript, it is not necessary to use semicolons at the end of
			statements in order for a program to function. However, leaving
			semicolons out of code can lead to innumerable problems, especially
			if a code condenser is used (which is common with JavaScript).
			Therefore, this style guide \emph{requires} the use of semicolons in
			all optional locations in JavaScript.
	\section{C++}
		Unless otherwise specified here, follow the guidelines of the
		\href{http://google-styleguide.googlecode.com/svn/trunk/cppguide.xml}{Google C++ Style Guide}.
		\subsection{Braces}
			\begin{itemize}
				\item Opening braces should always be on their own line, aligned
					on the column of the item they are for.
				\item Close braces should always be on their own line, aligned
					horizontally with their open statement.
			\end{itemize}

\chapter{Functional Programming}
	\section{All Languages}
		\subsection{Commenting}
			Comments should be included that give detailed descriptions of all functions,
			classes, and instance variables.
			\subsubsection{Documentation}
				When possible, DocBlock-style comments should be provided for
                \emph{every} class and method (public or private). DocBlock
                comments should be formatted using the \texttt{/** ... */}
                system, as illustrated in Figure \ref{lst:docblock-functional}
                on page \pageref{lst:docblock-functional}.

				The most commonly used DocBlock interpreter for C documentation
                is \href{http://www.stack.nl/~dimitri/doxygen/}{Doxygen}. It
                supports export into a multitude of formats and supports a large
                number of tags. For a reference of what tags are supported, see
                the
                \href{http://www.stack.nl/~dimitri/doxygen/manual/commands.html}{Special Commands}
				section of the
                \href{http://www.stack.nl/~dimitri/doxygen/manual/index.html}{Doxygen Manual}.

				Author and date (since) should \emph{always} be given for a
                class DocBlock. If using version numbering for the project, the
                version tag should also \emph{always} appear in the class
                DocBlock.

				At a minimum, functions/methods should \emph{always} have the
                date (since) tag. If multiple authors have contributed to the
                same class, they should also include the author tag. (If not
                given, the author is to be assumed to be the same as the
				class author.)

				Functions/methods should also have \texttt{@param} and
                \texttt{@return} tags where applicable.
				Any method other than a simple getter/setter (one which does not operate on the
				variable, but merely sets or returns its value) must have a description of its
				functionality. When applicable to the language, parameter tags should also
				specify the parameter direction (\texttt{[in]} or \texttt{[out]}). The description
				portion of these tags should be located on the next line (after the tag and variable
				name) and indented one level past the tag itself.

				Global variables and fields should have a DocBlock comment to explain their
				purpose, and macros should be described in a block somewhere near the
				beginning of the file.

				Description text should be written in sentences and be punctuated with a period.
				Variable and \texttt{@param}/\texttt{@return} value descriptions, however, should be kept
				brief and not written as full sentences, unless absolutely necessary. (The first
				letter of these descriptions, though, should still be capitalized.)

				A blank line should be placed after each paragraph of explanation; between the
				explanation and author, date, and version tags; and between those tags and the
				param and return tags.

				\begin{figure}[h!]
					\label{lst:docblock-functional}
					\caption{Example DocBlock-style Commenting}
					\lstset{language=C}
					\begin{lstlisting}
/**
 * This is an explanation of a function.
 *
 * @author Ethan Ruffing <ruffinge@gmail.com>
 * @since 2014-08-06
 *
 * @param[in] varName
 *     A brief description of the parameter varName
 * @param[out] outName
 *     A brief description of the output outName
 * @return
 *     A brief description of the function's return value
 */
					\end{lstlisting}
				\end{figure}
	\section{C}
		Unless otherwise specified here, follow the guidelines of the
		\href{http://google-styleguide.googlecode.com/svn/trunk/cppguide.xml}{Google C++ Style Guide},
		paying attention only to those portions applicable to plain C.
		\subsection{Braces}
			\begin{itemize}
				\item Opening braces should always be on their own line, aligned
                    on the column of the item they are for.
				\item Close braces should always be on their own line, aligned
                    horizontally with their open statement.
			\end{itemize}

\chapter{Markup Languages}
	\section{All Languages}
        Markup languages, by their nature, are largely self-documenting.
        Therefore, when writing in a markup language, it is permissible to be
        more lax when commenting than in other languages.
	\section{HTML}
		Unless otherwise specified here, follow the guidelines of the
		\href{http://www.w3.org/TR/html5/syntax.html}{W3C HTML5 Syntax Recommendations}.
		\subsection{Paragraphing}
			In writing HTML, there is occassionally a tendency to write multiple
            tags on own line. This should be avoided, as it leads to difficulty
            in reading the source. Instead, place each open and close tag on
            their own lines, with the contents of those tags indendented between
            them.

			The one exception to this rule is for trivial tag contents, such as
            the text in an \texttt{href}, in which case it is permissible (even
            advisable) to place the contents directly between the tags on the
            same line.
	\section{CSS}
		\subsection{Braces}
            When programming in CSS, follow the ``end-of-line'' opening brace
            style.
			\begin{itemize}
				\item Opening braces should always be on the same line as their
                    parent.
				\item Close braces should always be on their own line, aligned
                    horizontally with their open statement.
			\end{itemize}
		\subsection{spaces}
			A space should be placed after each colon (\texttt{:}), but
            \emph{not} before the colon.

\end{document}
